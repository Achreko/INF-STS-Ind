\documentclass[11pt]{article}
\usepackage[T1]{polski}
\usepackage{fancyhdr}
\usepackage{pdfpages}
\usepackage[utf8]{inputenc} 
\usepackage{hyperref}

 
\pagestyle{fancy}
\title{Keybored - sprawozdanie z projektu}
\author{Kacper Achramowicz}
\fancyhead[L]{Kacper Achramowicz\\
Projekt indywidualny "Gra zręcznościowa Keybored"}
\fancyhead[R]{16.VI.2021\\
V1.0}
\begin{document}

\maketitle

\tableofcontents
\clearpage

\section{Wstęp}
Zadaniem projektowym postawionym przede mną przez prowadzącego było napisanie gry mobilnej na system Android. Stworzona przeze mnie aplikacja to "Keybored", którą napisałem w Pythonie. Przy tworzeniu GUI korzystałem również z framerwoka kivy. Pliki *.apk tworzyłem korzystając z programu buildozer. Kod źródłowy umieściłem w repozytorium: \url{https://github.com/Achreko/INF-STS-Ind}
\section{Analiza funkcjonalna}
\subsection{Słownik}
\textbf{przegrana} - nienaciśniecię odpowiedniego przycisku w interwale, \\
\textbf{wygrana} - naciskanie odpowiednich przycisków do czasu, gdy interwał będzie mniejszy niż 0,5 sekundy, \\
\textbf{plansza} - ekran podzielony na 9 prostokątów równej wielkości, gdzie użytkownik rozgrywa partię, \\
\textbf{partia} - rozgrywka zaczynająca się od naciśnięcia przycisku start na pierwszym ekranie, lub po wybraniu opcji restart w wypadku wygranej lub przegranej polegająca na naciskaniu podświetlających się na biało przycisków w zmniejszających się interwałach\\
\textbf{gameloop} - dosłownie pętla gry, czyli powtarzające się akcje stanowiące grę
\subsection{Opis funkcjonalności}
Program spełnia początkowe założenia z wyjątkiem jednego, a mianowicie przycisku pozwalającego na wyłączenie aplikacji. Funkcjonalność ta okazała się niepotrzebna ze względu na specyfikę wyłączania aplikacji na urządzeniach z systemem Android. Do funkcjonalności należą:\\
- wyświetlanie ekranu startowego, \\
- przejście z ekranu startowego do planszy po naciśnięciu przycisku "Start", \\
- rozegranie partii, \\
- w wypadku przegranej wyświetlenie odpowiedniego pop-upu z przyciskiem "Restart",\\
- poinformowanie użytkownika o wygranej odpowiednim pop-upem, który również zawiera przycisk "Restart".
\section{Implementacja}
Aplikacja zawiera się w plikach manager.kv man.py oraz main.py. Główną trudnością implementacji było odpowiednie połączenie klas napisanych w Pythonie z regułami zapisanymi w kivy language.
\subsection{main.py}
Ustawia wymagania dotyczące posiadanej wersji kivy oraz łączy wszystkie pliki. Zawiera klasę Keybored(App) odpowiadającą za stworzenie i uruchomienie aplikacji.
\subsubsection{Metoda build(self)}
Zwraca główny widget(root\_widget), a następnie uruchamia aplikację.
\subsection{man.py}
Tworzy główny widget ładując do niego dane zawarte w pliku manager.kv oraz definiuje metody jego widgetów. Zawiera puste klasy, do których reguły zostały utworzone w kivy language, czyli klasy:\\
-Man(ScrenManager), która jest głównym widgetem,
-MenuScreen(Screen) czyli ekran startowy,
-Lost(Popup), czyli okienko przegranej,
-Won(Popup), czyli okienko wygranej.
Jedyna klasa posiadająca pola i metody w tym pliku to klasa \textbf{GameScreen(Screen)}. W ostatniej linii tego pliku wczytywany jest root\_widget z pliku manager.kv.
\subsubsection{Pole dt}
Długość interwału w sekundach.
\subsubsection{Pole ct}
Licznik naciśnięć poprawnego przycisku. Jeżeli nie jest równy polu \textit{it} to gracz przegrywa.
\subsubsection{Pole it}
Licznik wyświetleń przycisku. Jeżeli nie jest równy polu \textit{ct} to gracz przegrywa.
\subsubsection{Pole flag}
Flaga pozwalająca stwierdzić, czy można zacząć gameloop.
\subsubsection{Pole prop}
Numer przycisku, który należy nacisnąć.
\subsubsection{Pole timer}
Zegar realizujący interwał.
\subsubsection{Metoda restart(self)}
Przywraca wartości początkowe polom, przywraca planszę do stanu wejściowego.
\subsubsection{Metoda change\_flag(self)}
Zmienia wartość flag na True oraz wywołuje metodę game\_loop(self).
\subsubsection{Metoda game\_loop(self)}
Inicjuje pole timer.
\subsubsection{Metoda randomizer(self, interval)}
Korzystając z generatora liczb pseduolosowych podświetla odpowiedni przycisk.
\subsubsection{Metoda check(self, instance)}
Metoda wywoływana po naciśnieciu przycisku, sprawdza, czy został naciśnięty odpowiedni przycisk wywołując przy tym metodę get\_id(self, instance) i jeżeli ten warunek nie zostanie spełniony to wywołuje metodę loss(self). Sprawdza też, czy nie nastąpiła wygrana. Jeżeli tak to zatrzymuje timer i wywołuje pop-up Won(Popup).
\subsubsection{Metoda get\_id(self, instance)}
Ustala, który przycisk został naciśnięty.
\subsubsection{Metoda loss(self)}
Zatrzymuje timer i wywołuje pop-up Loss(Popup).
\subsection{manager.kv}
W pliku wyznaczam root\_widget oraz podpinam do niego pozostałe widgety.\\ Definiuję również etykietę <GameButton@ToggleButton>, czyli przycisku służące do gry bazujące na przycisku typu ToggleButton. W etykiecie nadaję przyciskom kolor oraz to, że po naciśnięciu wywoływana jest metoda man.gamescreen.check(self, instance).\\
Nadaję reguły opisujące pop-upy <Lost> oraz <Won> odpowiadające kolejno klasom Loss(Popup) i Won(Popup), które robią w praktyce to samo różniąc się jedynie podpisem oraz podpinam do nich przycisk "Restart".\\
Tworzę reguły opisujące <MenuScreen> odpowiadający klasie\\ man.menuscreen(Screen).\\
Do <GameScreen> odpowiadającemu klasie man.gamescreen(Screen) dołączam 9 instancji etykiety <GameButton>.
\section{Analiza i wnioski}
Analizując wykonaną pracę stwierdzam, że mogłoby być dobrą praktyką przeniesienie pustych klas do oddzielnych plików, ale są też argumenty przeczące temu stwierdzeniu. Musiałbym wtedy w każdym z plików importować from kivy.lang import Builder, aby wczytywać reguły zapisane w kivy language z ciągu znaków, a nie z pliku. Mogłoby to zajmować więcej czasu i pamięci.\\
Kolejnym wnioskiem jest to, że w celu udoskonalenia GUI można skorzystać jeszcze z kivy material design (kivymd), gdzie znajduje się wiele preconfigurowanych widgetów. W tym projekcie postanowiłem skorzystać jednak z tych utworzonych ręcznie w celu lepszego zrozumienia ich działania.\\
Wniosek ostatni, a dla mnie jako studenta najważniejszy to jak ważna jest umiejętność czytania dokumentacji. Zaczynając projekt pierwsze reguły w kivy language tworzyłem na przestrzeni dni, gdzie po dokładniejszym zapoznaniu się z dokumentacją problemy, które na początku uznałem za niewarte rozwiązywania ze względu na potrzebny do nich nakład pracy i ich stosunkowy brak wagi, rozwiązałem w ostatnim dniu pisania projektu.
\end{document}