\documentclass[11pt]{article}
\usepackage[T1]{polski}
\usepackage{graphicx}
\usepackage{xcolor}
\usepackage{listings}
\usepackage{fancyhdr}
\usepackage{pdfpages}
\usepackage[utf8]{inputenc} 

 
\pagestyle{fancy}
\fancyhead[L]{Kacper Achramowicz\\
Projekt indywidualny IV semestr}
\fancyhead[R]{15.V.2021\\
V1.0}
\begin{document}

\begin{huge}
\begin{center}
\textbf{Specyfikacja funkcjonalna}
\end{center}
\end{huge}

\tableofcontents

\section{Opis ogólny}
Progrram nazywa się Key-Bored. Jest to gra zręcznociowa na system Android, która polega na naciskaniu odpowiednich części ekranu w zmniejszających się interwałach czasowych.

\section{Ogólna funkcjonalność}
\subsection{Korzystanie z programu}
Program wykonany jest w formie aplikacji mobilnej na systemy Android. Do jej włączenia wymagane jest posiadanie systemu Android lub jego emulatora.
\subsection{Możliwości programu}
Aplikację można włączyć uruchamiając plik wykonywalny \textbf{Key-Bored.exe}. Możliwości programu to włączenie nowej sesji gry i podświetlanie odpowiednich fragmentów planszy. Program może zmniejszać interwały czasowe i determinować, kiedy gracz przegra. W takim przypadku program umożliwia restart sesji gry.

\section{Format danych i struktura plików}
\subsection{Słownik}
Plansza - ekran urządzenia z systemem Android\\
Gracz - inaczej użytkownik\\
Sesja - rozgrywka zaczynająca się od wciśnięcia przycisku start do wygranej lub przegranej
\subsection{Struktura katalogów}
Wszystkie pliki potrzebne do skompilowania pliku wykonywalnego będą w katalogu \textit{src}. Wszystkie testy znajdować się będą w katalogu \textit{test}.
\subsection{Przechowywanie danych w programie}
Dane w programie plansza będzie przechowywana w postaci listy obiektów, które będą miały atrybuty odpowiadające za kolor i wielkość.
\section{Sceniariusz działania programu}
\subsection{Włączanie programu}
\subsection{Wyświetlenie menu głównego}
\subsection{Wybranie opcji start}
\subsection{Wygranie lub przegranie sesji}
\subsection{Wyłączenie programu}


\section{Testowanie}
Do testów kodu użyty został framework testowy Pytest. Testy mają na celu zmniejszenie szansy niespodziewanego zachowania się programu w przypadkach skrajnych. GUI testowane będzie w sposób empiryczny.

\end{document}